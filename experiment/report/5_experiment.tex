\section{Experiment}
\label{sec:experiment}

This section outlines the experimental strategy used to address the research question RQ1 presented in section~\ref{sec:introduction}, following the methodology detailed in section~\ref{benchmarking_design}. Our approach proceeds in two global phases. We first establish the baseline for the vTP algorithm by detailing its mathematical formulation and comparing the resulting quantum state with the results derived from running the circuit on a noiseless Qiskit Aer simulator. Having established the foundation, we advance to the core experiment, which is structured in three sequential stages aiming to identify correlations between pre-runtime metrics such as gate count, circuit depth, and payload size, with the success rate.

An important aspect for this entire experimental strategy is the restriction of all randomized operations to gates within the Clifford group. The rationale for this restriction is based on the well-established and classifiable mathematical structure of the group \cite{grierClassificationCliffordGates_2022}. A primary advantage is their inherent efficient simulation on classical hardware; unlike universal quantum circuits, Clifford-based circuits can be simulated on classical computers in polynomial time, which allows for direct corroboration of experimental outputs against a noiseless, theoretical framework. In addition, these gates are integral to quantum error correction, underscoring the importance of their characterization in the advancement of fault-tolerant systems. Because of this, the use of gates within the Clifford group provides a controlled and manageable environment for the development of robust benchmarking methodologies.

\subsection{Experimental Hypotheses}
Based on our preliminary review of quantum runtime services, we formulate the following hypotheses:

\begin{itemize}
\item \label{h1} \textbf{H1:} The number of single qubit gates will have a minimal impact on the final success rate.
\item \label{h2} \textbf{H2:} There exists a negative correlation between the depth of the circuit and the success rate.
\item \label{h3} \textbf{H3:} The payload size emerges as a significant factor correlated with the success rate.
\end{itemize}

\subsection{Stage 1: Preparation}

Initially, the Qiskit Aer simulator was employed to assess the alignment between the vTP's theoretical forecasts and the simulation results. Figure~\ref{fig:vtp_circuit} illustrates the fundamental circuit, while comprehensive details can be found in \ref{sec:vtp}.

The experimental setup was designed to execute jobs consisting of 1024 shots each. Upon aggregation of the results, a 100\% success rate was observed, which is expected since the simulator did not include any noise model. The primary objective was to measure qubit R after the validation sequence, with an anticipated result of \(\ket{0}\). In particular, all the jobs yielded a count of \(\{\text{'0': 1024}\}\).

To compare the results of the simulator with the theoretical analysis, a four-qubit system \((\ket{RMAB})\) should be considered at key stages of circuit execution, as detailed below.

\begin{itemize}
    \item The quantum state to be teleported, represented by the \texttt{'after\_payload'} barrier in figure~\ref{fig:teleportation_algorithm}), corresponds to the state vector:
    \begin{equation}
    \ket{\psi_{\text{ap}}} = \frac{\sqrt{2}i}{2} (\ket{0001} - \ket{0010})
    \end{equation}
    The state can be written as \(|0_R 0_M\rangle \otimes \frac{i}{\sqrt{2}}(\ket{0_A 1_B} - \ket{1_A 0_B})\).
    \item After applying the vTP protocol, at the barrier \textit{"before validation"}, the state vector becomes: 
    \begin{equation}
    \begin{split}
    \ket{\psi_{\text{bv}}} &= \frac{\sqrt{2}i}{4} (\ket{0001} + \ket{0011} + \ket{0101} \\ 
    & + \ket{0111} - \ket{1000} - \ket{1010} \\ 
    & - \ket{1100} - \ket{1110})
    \end{split}
    \end{equation}
    \item Finally, after the validation sequence, at \textit{"after\_validation"} barrier, the system's state vector is:
    \begin{equation}
    \ket{\psi_{\text{av}}} = \frac{1}{2} (\ket{0000} + \ket{0010} + \ket{0100} + \ket{0110})
    \end{equation}
\end{itemize}

Note that the state can be written as \( \ket{0_R} \otimes \frac{1}{\sqrt{2}}(\ket{0_M} + \ket{1_M}) \otimes \frac{1}{\sqrt{2}}(\ket{0_A} + \ket{1_A}) \otimes \ket{0_B} \), which simplifies to \( \ket{0_R} \ket{+_M} \ket{+_A} \ket{0_B} \).

The final state \(\ket{\psi_{\text{av}}}\) shows that qubit R is in the state \(\ket{0}\). For this reason, measurements of qubit R invariably result in '0', thereby confirming the successful implementation and verification of the vTP. This outcome is consistent with the success rate 100\% observed in the simulation.

\subsubsection{Comparing simulation with math analysis} 

The simulator data mirror the step-by-step state evolution displayed in Section~\nameref{sec:vtp}. In that section, the conceptual state is written for the three logical qubits \((M,A,B)\), whereas the circuit register is ordered \((R,M,A,B)\). Dropping the leftmost bit of every four-bit computational basis label gives us a state that must be compared with Equations~\ref{eq:teleportation_grouped_state}--\ref{eq:final_state_variation}:

\begin{itemize}
    \item \textbf{After payload.} Removing R from the simulated vector \(\frac{\sqrt{2}i}{2}(\ket{0001}-\ket{0010})\) produces \(\frac{i}{\sqrt{2}}(\ket{001}-\ket{010})_{MAB}\), which matches the anti-symmetric Bell component obtained immediately after the payload gate in the theoretical result.
    \item \textbf{Before validation.} Tracing R in \(\ket{\psi_{\text{bv}}}\) reproduces the superposition of Eq.~\ref{eq:state_after_cx_ab}, the state predicted after the sequence \(CX_{AB}\) followed by \(CZ_{MB}\).
    \item \textbf{After validation.} Discarding R from \(\ket{\psi_{\text{av}}}\) yields \(|+\rangle_M|+\rangle_A|0\rangle_B\), exactly Eq.~\ref{eq:final_state_variation}. Hence, Bob's qubit B holds the original message state, while M and A are measured in the \(|+\rangle\) state and are disentangled from B, as predicted by theory.
\end{itemize}

Based on these observations, it is evident that the practical implementation aligns accurately with the predictions of the analytical model.

\subsection{Stage 2: N random gates experiment}
\label{subsec:nrandom}

Having confirmed the correctness of the vTP algorithm, we proceeded to the second stage to validate our first hypothesis (\ref{h1}). This stage involved a stress test targeting the circuit's reliability on the IBM Quantum runtime service (ibm\_sherbrooke) and the Rigetti Ankaa-3 through qBraid and the AWS quantum service. The experiment was designed by inserting a progressively increasing number of single qubit operations into the payload, with the total number of gates ranging from $0$ to $20,000$ and the payload size varying from $1$ to $5$ qubits.

To handle the extensive parameter space, jobs were executed in batches, each handling specific ranges of random gates on IBM hardware: $(200, 205)$, $(500, 505)$, $(1000, 1005)$, $(1500, 1505)$, $(2000, 2005)$, $(3000, 3005)$, $(5000, 5005)$, $(10000, 10005)$, $(20000, 20005)$. However, the Rigetti Ankaa-3 has a gate limit of 20k, therefore, on that hardware, the maximum range is limited to $(7000, 7005)$. The motivation for these ranges is to perform identical tasks with nearby gate counts while enabling the distinction between ranges.

It should be noted that for this experiment, we considered the inclusion of random unitary gates in the payload that were not restricted to the Clifford group. Due to this, their conjugate operations, required for validation, were not necessarily straightforward to implement and might not align with simple Pauli operations. In addition, the presence of quantum states beyond the scope of Pauli operations could result in the introduction of additional errors during measurement, as elaborated in Section ~\ref{sec:background}. That is why, prior to the execution of the second experiment, the codebase was adjusted to generate exclusively random gates within the Clifford group.

\begin{table}[ht]
\centering
\caption{Success Rate Statistics for vTP on IBM Sherbrooke Across Different Gate Count Ranges}
\label{tab:vtp_success_rates_ibm}
\begin{tabular}{|c|c|c|}
    \hline
    \textbf{Gate Count Range} & \textbf{Success Rate Range (\%)} & \textbf{Mean (\%)} \\
    \hline
    10 & 2.0--87.4 & 32.7 \\
    \hline
    30 & 2.3--42.6 & 15.5 \\
    \hline
    50 & 2.1--17.8 & 7.3 \\
    \hline
    70 & 2.1--12.5 & 5.0 \\
    \hline
    200 & 2.0--73.3 & 25.9 \\
    \hline
    500 & 5.1--64.0 & 23.5 \\
    \hline
    1,000 & 5.8--48.7 & 23.4 \\
    \hline
    1,500 & 5.4--53.1 & 24.0 \\
    \hline
    3,000 & 5.1--53.1 & 25.6 \\
    \hline
    5,000 & 5.7--53.8 & 24.6 \\
    \hline
    10,000 & 5.6--50.4 & 23.5 \\
    \hline
    20,000 & 3.7--52.9 & 22.9 \\
    \hline
\end{tabular}
\end{table}

\begin{table}[ht]
\centering
\caption{Success Rate Statistics for vTP on Rigetti Ankaa-3 Across Different Gate Count Ranges}
\label{tab:vtp_success_rates_rigetti}
\begin{tabular}{|c|c|c|}
    \hline
    \textbf{Gate Count Range} & \textbf{Success Rate Range (\%)} & \textbf{Mean (\%)} \\
    \hline
    10 & 0.0--90.0 & 52.2 \\
    \hline
    30 & 0.0--60.0 & 24.3 \\
    \hline
    50 & 0.0--40.0 & 10.8 \\
    \hline
    70 & 0.0--30.0 & 8.3 \\
    \hline
    500 & 0.0--90.0 & 40.8 \\
    \hline
    5,000 & 10.0--90.0 & 39.5 \\
    \hline
\end{tabular}
\end{table}

Tables~\ref{tab:vtp_success_rates_ibm} and~\ref{tab:vtp_success_rates_rigetti} summarize the success rate statistics for vTP after executing circuits with different gate count ranges in IBM Sherbrooke and Rigetti Ankaa-3, respectively. Figure~\ref{fig:random_gates_success_rate} illustrates that while overall success rates remain low and highly variable, a declining trend is observed in gate ranges of \ensuremath{\leq} 70.

\begin{figure*}[ht]
    \centering
    \begin{subfigure}[b]{0.48\textwidth}
        \centering
        \includegraphics[width=\linewidth]{img/2b_success_rate_vs_gates_grouped_boxplot_filtered.png}
        \caption{All gate count ranges}
        \label{fig:random_gates_success_rate_boxplot}
    \end{subfigure}\hfill
    \begin{subfigure}[b]{0.48\textwidth}
        \centering
        \includegraphics[width=\linewidth]{img/2b_filtered_success_rate_correlation_gates.png}
        \caption{Filtered view (gates \ensuremath{\leq} 70)}
        \label{fig:filtered_gates_success_rate}
    \end{subfigure}
    \caption{Success rate analysis with respect to number of gates. (a) Box plot comparison across all gate count ranges. (b) Filtered correlation view focusing on lower gate counts (\ensuremath{\leq} 70).}
    \label{fig:random_gates_success_rate}
\end{figure*}

\begin{table}[ht]
\centering
\caption{Correlation statistics between number of gates and success rate for gate counts $\leq 70$.}
\label{tab:gate_correlation_stats}
\begin{tabular}{lcc}
\hline
\textbf{Statistic} & \textbf{IBM Sherbrooke} & \textbf{Rigetti Ankaa-3} \\
\hline
Pearson r          & -0.467 & -0.558 \\
Spearman $\rho$    & -0.554 & -0.568 \\
$R^2$              & 0.218  & 0.311  \\
Slope              & -0.007 & -0.009 \\
p-value            & $6.04 \times 10^{-17}$ & $5.09 \times 10^{-16}$ \\
\hline
\end{tabular}
\end{table}

Table~\ref{tab:gate_correlation_stats} provides statistical evidence for the trend suggested in Figure~\ref{fig:random_gates_success_rate}: both platforms exhibit a moderate inverse relationship between gate count and success rate in the lower gate ranges, with the effect being more pronounced for Rigetti. 

Following the ASA definition, the p-value represents the probability, under a specified statistical model that assumes no correlation, of observing the given data \cite{Wasserstein02042016_2016}. In this context, the calculated p-values (\ensuremath{p < 10^{-15}} for both platforms) strongly suggest a mismatch between the observed data and the null hypothesis of zero correlation, indicating that a correlation exists.

Figure~\ref{fig:success_rate_gates_by_payload} presents a segmented analysis that separates the data by payload size for both platforms. This analysis reveals that the payload size is the dominant factor that influences the success rate and, at the same time, makes clear that the gate count has only a modest effect (10 - 20\%) in the success rate within each category. For instance, on IBM, circuits with 1-qubit payloads maintain relatively high success rates (\textasciitilde50-60\%) regardless of gate count, while 5-qubit payloads consistently exhibit low success rates (\textasciitilde5\%) across all experiments.

\begin{figure*}[ht]
    \centering
    \begin{subfigure}[b]{0.48\textwidth}
        \centering
        \includegraphics[width=\linewidth]{img/2d_success_rate_vs_gates_by_payload_ibm.png}
        \caption{IBM Sherbrooke}
        \label{fig:success_rate_gates_by_payload_ibm}
    \end{subfigure}\hfill
    \begin{subfigure}[b]{0.48\textwidth}
        \centering
        \includegraphics[width=\linewidth]{img/2d_success_rate_vs_gates_by_payload_rigetti.png}
        \caption{Rigetti Ankaa-3}
        \label{fig:success_rate_gates_by_payload_rigetti}
    \end{subfigure}
    \caption{Success rate versus gate count segmented by payload size for IBM Sherbrooke and Rigetti Ankaa-3. Each line represents a different payload size, with shaded regions indicating standard deviation. The x-axis is in Log-scale to enables visualization across the wide gate range tested.}
    \label{fig:success_rate_gates_by_payload}
\end{figure*}

The experimental results at this stage partially support our initial hypothesis (\ref{h1}), though the observed correlation is stronger than expected. While we hypothesized a minimal impact of gate count, the data reveal a moderate negative correlation ($R^2 = 0.218$ for IBM and $R^2 = 0.311$ for Rigetti) between gate count and success rate in the lower gate ranges. However, when the payload size increases, the success rates drop sharply regardless of the gate count, confirming that the payload size remains the dominant limiting factor. This pattern holds across both hardware platforms, reinforcing that the gate count effect, while measurable, remains secondary compared to the payload size effect.

The moderate gate count correlation observed can be attributed to the cumulative effect of imperfect single-qubit gate operations, though these operations generally display high fidelity in today's quantum hardware \cite{shuklaCompleteCharacterizationDirectlya_2020}. The fact that the correlation becomes evident only at lower gate counts ($\leq 70 gates$) suggests that single-qubit gate errors accumulate measurably but are quickly overshadowed by the more substantial errors introduced by two-qubit entangling operations required for larger payload sizes. Data supporting this section can be accessed at \url{https://github.com/xthecapx/qc_experiment/tree/main/experiment/ieee_analysis}.

\subsection{Stage 3: Circuit depth experiment}

To analyze the circuit depth, a function was developed that receives as input parameter a circuit depth value and returns as a result a list of pairs of \texttt{payload\_size} and \texttt{num\_gates} that can be used to generate a payload that combined with the required gates of vTP produces a given depth. For example, to get a depth of 13 for the vTP, the only possible combination is to use a \texttt{payload\_size} of 1 with a single gate.%, because with that setup the final depth of the vTP is 13.

The function uses a simple linear model, as defined in equation \ref{eq:base_depth}, to generate the list of pairs. Since this research aims to determine the correlations with the success rates, rather than developing an optimal model for generating circuits with specific depths, the simple model serves the purpose.

\begin{equation}
\text{base\_depth} = 13 + 2 \times (\text{payload\_size} - 1)
\label{eq:base_depth}
\end{equation}

The \texttt{payload\_size} formula was derived empirically through the process of experimenting with vTP. As we explain before, the constant 13 represents the minimum circuit depth observed for a single-qubit payload (\texttt{payload\_size} = 1) with a baseline X-gate operation. The linear term $(2 \times (\text{payload\_size} - 1))$ corresponds to the additional circuit depth required for each additional qubit in the payload, where approximately two depth units are added per qubit due to entanglement operations and random gate operations. This empirical model serves as a simple predictor for circuit depth estimation in preparing experiments with different payload configurations.

Following this, the \texttt{target\_depth} is defined by taking into account the mean number of supplementary random gates for each payload qubit:

\begin{equation}
\begin{split}
&\text{target\_depth} = \\ 
&\text{base\_depth} + 2 \times \left(\frac{\text{num\_gates}}{\text{payload\_size}}\right) - 2
\end{split}
\label{eq:target_depth_model}
\end{equation}

Within equation~\ref{eq:target_depth_model}, the relation between the number of gates and the payload size serves to estimate the mean depth introduced by the random Clifford gates applied to the payload. The subtraction of 2 accounts for instances where random gates (\texttt{num\_gates} > 0) are deployed, supplanting the default X-gate (nominal depth of 1, yet effectively 2 when the validation sequence is considered) on each payload qubit whose depth contribution was included as part of \texttt{base\_depth}. To determine \text{number\_gates} based on predetermined \texttt{target\_depth} and \texttt{payload\_size}, Equation~\ref{eq:target_depth_model} is adjusted accordingly.

\begin{equation}
\begin{split}
& \text{num\_gates} = \\ 
&\frac{(\text{target\_depth} - \text{base\_depth} + 2) \times \text{payload\_size}}{2}
\end{split}
\label{eq:required_gates}
\end{equation}

Observe that equation~\ref{eq:required_gates} is capable of generating floating-point numbers; consequently, within the code, these values are cast to integers, which results in the truncation of the decimal component without the application of any rounding operations. Furthermore, it is important to acknowledge that \texttt{base\_depth} may potentially exceed \texttt{target\_depth}; therefore, the code imposes a restriction to only incorporate positive values.

It is worth noting that the valid configurations were chosen exclusively when this calculation resulted in a non-negative integer for \texttt{num\_gates}. This methodology enabled the systematic generation of randomized circuits, varying in \texttt{payload\_size} and \texttt{num\_gates}, while maintaining consistent operational depth. However, although the calculations were executed flawlessly within the simulator, the final circuit depth changed during the transpilation phase and by the intrinsic circuit optimization of the quantum processor before the execution of the algorithm.

\begin{table}[ht]
\centering
\caption{Success Rate Statistics for vTP on IBM Sherbrooke Across Different Circuit Depth Ranges}
\label{tab:vtp_success_rates_depth_ibm}
\begin{tabular}{|c|c|c|}
    \hline
    \textbf{Circuit Depth Range} & \textbf{Success Rate Range (\%)} & \textbf{Mean (\%)} \\
    \hline
    5--9 & 58.5--87.4 & 66.6 \\
    \hline
    10--14 & 6.6--86.5 & 40.5 \\
    \hline
    15--19 & 2.0--68.2 & 13.7 \\
    \hline
    20--24 & 2.3--69.2 & 22.3 \\
    \hline
    25--29 & 2.7--68.3 & 23.1 \\
    \hline
    30--34 & 2.1--69.1 & 24.7 \\
    \hline
    35--39 & 2.1--69.2 & 22.2 \\
    \hline
    40--44 & 2.5--69.3 & 23.7 \\
    \hline
\end{tabular}
\end{table}

\begin{table}[ht]
\centering
\caption{Success Rate Statistics for vTP on Rigetti Ankaa-3 Across Different Circuit Depth Ranges}
\label{tab:vtp_success_rates_depth_rigetti}
\begin{tabular}{|c|c|c|}
    \hline
    \textbf{Circuit Depth Range} & \textbf{Success Rate Range (\%)} & \textbf{Mean (\%)} \\
    \hline
    10--14 & 70.0--100.0 & 84.0 \\
    \hline
    15--19 & 10.0--90.0 & 51.3 \\
    \hline
    20--24 & 0.0--100.0 & 33.5 \\
    \hline
    25--29 & 0.0--90.0 & 30.4 \\
    \hline
    30--34 & 0.0--100.0 & 31.0 \\
    \hline
    35--39 & 0.0--90.0 & 33.0 \\
    \hline
    40--44 & 0.0--90.0 & 31.5 \\
    \hline
    45--49 & 0.0--100.0 & 32.3 \\
    \hline
\end{tabular}
\end{table}

Tables~\ref{tab:vtp_success_rates_depth_ibm} and~\ref{tab:vtp_success_rates_depth_rigetti} summarize the success rate statistics for vTP after executing circuits with different circuit depth ranges in IBM Sherbrooke and Rigetti Ankaa-3, respectively. The data reveal a correlation between circuit depth and success rate at lower depths: for IBM Sherbrooke, success rates drop from 66.6\% (depth 5--9) to 13. 7\% (depth 15--19), while for Rigetti Ankaa-3, the decline occurs from 84.0\% (depth 10--14) to 33. 5\% (depth 20--24).

Figure~\ref{fig:circuit_depth_success_rate_analysis} presents these data through two complementary visualizations: Subfigure~\ref{fig:circuit_depth_boxplot} shows the complete dataset across all depth ranges, while Subfigure~\ref{fig:filtered_depth_correlation} provides a filtered view focusing on the lower depth ranges where the correlation is most evident. To statistically validate this observed correlation, Table~\ref{tab:depth_correlation_stats} presents comprehensive correlation statistics that include Pearson and Spearman coefficients, values $R^{2}$, and p-values that demonstrate the statistical significance of the relationship between the depth of the circuit and the success rate.

\begin{figure*}[ht]
    \centering
    \begin{subfigure}[b]{0.48\textwidth}
        \centering
        \includegraphics[width=\linewidth]{img/3b_success_rate_vs_circuit_depth_boxplot.png}
        \caption{Complete dataset across all depth ranges}
        \label{fig:circuit_depth_boxplot}
    \end{subfigure}\hfill
    \begin{subfigure}[b]{0.48\textwidth}
        \centering
        \includegraphics[width=\linewidth]{img/3c_filtered_success_rate_correlation_circuit_depth.png}
        \caption{Filtered view (IBM: depths 5--19, $n=171$; Rigetti: depths 10--24, $n=60$)}
        \label{fig:filtered_depth_correlation}
    \end{subfigure}
    \caption{Success rate correlation analysis for circuit depth experiment. (a) Box plot comparison showing all depth ranges. (b) Filtered envelope plot focusing on lower depth ranges.}
    \label{fig:circuit_depth_success_rate_analysis}
\end{figure*}

\begin{table}[ht]
\centering
\caption{Correlation statistics between circuit depth and success rate at lower depths.}
\label{tab:depth_correlation_stats}
\begin{tabular}{lcc}
\hline
\textbf{Statistic} & \textbf{IBM Sherbrooke} & \textbf{Rigetti Ankaa-3} \\
\hline
Pearson r          & -0.714 & -0.538 \\
Spearman $\rho$    & -0.761 & -0.532 \\
$R^2$              & 0.510  & 0.289  \\
Slope              & -0.053 & -0.049 \\
p-value            & $6.04 \times 10^{-28}$ & $9.50 \times 10^{-6}$ \\
\hline
\end{tabular}
\end{table}

Table~\ref{tab:depth_correlation_stats} presents the statistical analysis of the correlation between the depth of the circuit and the success rate for the lower circuit depth ranges. The negative Pearson and Spearman correlation coefficients confirm a moderate inverse monotonic relationship, indicating that as the circuit depth increases, the success rate tends to decrease. In particular, IBM shows a stronger correlation ($r = -0.714$) compared to Rigetti ($r = -0.538$), which aligns with the visual patterns observed in Figure~\ref{fig:filtered_depth_correlation}.

Similarly to gate count analysis, the corresponding p-values (\ensuremath{< 10^{-5}} for both platforms) indicate a strong incompatibility between the observed data and the null hypothesis of no correlation, supporting the interpretation that deeper circuits degrade success rate before payload size effects dominate.

Figure~\ref{fig:success_rate_depth_by_payload} presents a segmented analysis that separates the data by payload size for both platforms, confirming that payload size is more dominant than circuit depth in influencing success rates. The segmented analysis reveals that while the circuit depth shows some correlation at lower depths (as previously reported), this effect diminishes when the payload size is controlled. For instance, on IBM, circuits with 1-qubit payloads maintain relatively high success rates (\textasciitilde65\%) regardless of circuit depth, while 5-qubit payloads consistently exhibit low success rates (\textasciitilde3\%) across all experiments.

\begin{figure*}[ht]
    \centering
    \begin{subfigure}[b]{0.48\textwidth}
        \centering
        \includegraphics[width=\linewidth]{img/3d_success_rate_vs_depth_by_payload_ibm.png}
        \caption{IBM Sherbrooke}
        \label{fig:success_rate_depth_by_payload_ibm}
    \end{subfigure}\hfill
    \begin{subfigure}[b]{0.48\textwidth}
        \centering
        \includegraphics[width=\linewidth]{img/3d_success_rate_vs_depth_by_payload_rigetti.png}
        \caption{Rigetti Ankaa-3}
        \label{fig:success_rate_depth_by_payload_rigetti}
    \end{subfigure}
    \caption{Success rate versus circuit depth segmented by payload size for IBM Sherbrooke and Rigetti Ankaa-3. Each line represents a different payload size, with shaded regions indicating standard deviation.}
    \label{fig:success_rate_depth_by_payload}
\end{figure*}

As a conclusion for stage 3, our findings support the second hypothesis \ref{h2}. There is a moderate negative correlation between the depth of a quantum circuit and the success rate. This behavior can be explained since the deeper circuits, which require more computational steps, are particularly susceptible to time-dependent errors, thus leading to a degradation in reliability. Comparable findings were documented by Perez et al. in \cite{perez_antonReliabilityIBMsPublic_2025}.

\subsection{Stage 4: Payload size experiment}

In the final experimental stage, we investigate the impact of increasing the payload size. A critical aspect of this stage is that each additional qubit added to the payload must be entangled with the existing message qubits to form a cohesive quantum state. In our experimental setup, this is achieved by adding a Controlled-NOT gate for each new payload qubit. As a result, this stage is designed not only to study the effect of adding more qubits but, more importantly, to analyze the impact of the corresponding linear increase in high-error two-qubit entangling operations. Also note that for each payload qubit, three stochastic unitary gates from the Clifford group were also incorporated.

Figure~\ref{fig:payload_success_rate_analysis} presents the success rate analysis for Stage 4 using two complementary visualizations. The boxplot in Figure~\ref{fig:payload_success_rate_boxplot} shows a negative trend in success rates as payload size increases. The filtered correlation analysis in Figure~\ref{fig:filtered_payload_correlation} quantifies this relationship, showing a strong negative correlation with payload size for both platforms.

\begin{figure*}[ht]
    \centering
    \begin{subfigure}[b]{0.48\textwidth}
        \centering
        \includegraphics[width=\linewidth]{img/1b_success_rate_vs_payload_size_combined_boxplot.png}
        \caption{Success rate distribution by payload size.}
        \label{fig:payload_success_rate_boxplot}
    \end{subfigure}\hfill
    \begin{subfigure}[b]{0.48\textwidth}
        \centering
        \includegraphics[width=\linewidth]{img/1c_filtered_success_rate_correlation_payload.png}
        \caption{Filtered success rate correlation vs payload size.}
        \label{fig:filtered_payload_correlation}
    \end{subfigure}
    \caption{Success rate analysis as a function of payload size for IBM Sherbrooke and Rigetti Ankaa-3. (a) Box plots comparing success rate distributions by payload size across both platforms. (b) Filtered correlation plot with mean success rate envelope, demonstrating a strong negative correlation between payload size and success rate.}
    \label{fig:payload_success_rate_analysis}
\end{figure*}

Table~\ref{tab:payload_correlation_stats} presents the correlation statistics for payload size versus success rate. Both platforms exhibit strong negative correlations, IBM showing a Pearson correlation coefficient of -0.877 ($R^2 = 0.770$) and Rigetti showing -0.831 ($R^2 = 0.690$). As explained previously in Stage 2, the p-values ($p < 0.001$) indicate that these correlations are highly statistically significant, providing strong evidence that the payload size is the dominant factor affecting the vTP success rates.

\begin{table}[ht]
\centering
\caption{Correlation statistics between payload size and success rate for payload sizes 1-5.}
\label{tab:payload_correlation_stats}
\begin{tabular}{lcc}
\hline
\textbf{Statistic} & \textbf{IBM Sherbrooke} & \textbf{Rigetti Ankaa-3} \\
\hline
Pearson r          & -0.877 & -0.831 \\
Spearman $\rho$    & -0.950 & -0.843 \\
$R^2$              & 0.770  & 0.690  \\
Slope              & -0.144 & -0.174 \\
p-value            & $3.13 \times 10^{-156}$ & $7.53 \times 10^{-59}$ \\
\hline
\end{tabular}
\end{table}

The results of Stage 4 confirm our third hypothesis \ref{h3}. There is a strong negative correlation between payload size and success rate, which establishes payload size as the most significant factor that influences circuit reliability. The p-values for payload size ($3.13 \times 10^{-156}$ for IBM and $7.53 \times 10^{-59}$ for Rigetti) provide the strongest statistical evidence among all stages, being substantially more extreme than those observed for gate count (Stage 2: $6.04 \times 10^{-17}$ and $5.09 \times 10^{-16}$) and circuit depth (Stage 3: $6.04 \times 10^{-28}$ and $9.50 \times 10^{-6}$). This analysis confirms that the degradation arises not just from including new qubits in the circuit but from the necessary inclusion of lower-fidelity CNOT operations required to entangle the message qubits within the payload.

\subsection{Final considerations}

In the last part of our experiment, we want to analyze the main component of the vTP protocol according to our findings. Implementing vTP without a payload requires two Hadamard gates, three CNOT gates, one swap operation, and a measurement. Single-qubit operations minimally affect the cumulative error;however, the three CNOT operations significantly contribute to errors. Additionally, read-out error must be considered as it may further distort the output quantum state. As a result, vTP stacks up moderate error in the NISQ era, which explains why useful data are obtained only from payloads of less than three qubits.

Tables~\ref{tab:vtp_execution_summary_ibm} and \ref{tab:vtp_execution_summary_rigetti} compile the workload breakdown for each experimental stage. Stages 1, 2, and 4 share the same data set (all experiments in all payload sizes), while stage 3 uses a filtered subset limited to circuit depths $\leq 50$ for focused analysis. The shot counts are reconstructed from measurement count dictionaries returned by each platform.

\begin{table}[ht]
\centering
\caption{IBM Sherbrooke execution summary.}
\label{tab:vtp_execution_summary_ibm}
\begin{tabular}{lrr}
\hline
\textbf{Stage} & \textbf{Jobs} & \textbf{Shots} \\
\hline
1. Hardware baseline & 485 & 1{,}986{,}443 \\
2. Gate count & 485 & 1{,}986{,}443 \\
3. Circuit depth & 291 & 1{,}191{,}936 \\
4. Payload size & 485 & 1{,}986{,}443 \\
\hline
\end{tabular}
\end{table}

\begin{table}[ht]
\centering
\caption{Rigetti Ankaa-3 execution summary.}
\label{tab:vtp_execution_summary_rigetti}
\begin{tabular}{lrr}
\hline
\textbf{Stage} & \textbf{Jobs} & \textbf{Shots} \\
\hline
1. Hardware baseline & 225 & 2{,}260 \\
2. Gate count & 225 & 2{,}260 \\
3. Circuit depth & 187 & 1{,}870 \\
4. Payload size & 225 & 2{,}260 \\
\hline
\end{tabular}
\end{table}
