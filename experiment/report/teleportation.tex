to reference: \cite{sprangerUnderstandingQuantumTeleportation_2023}

Understanding the Quantum Teleportation Protocol

Abstract – Decomposing a complex system into smaller abstract functional blocks and developing mathematical models to represent their behavior is an important activity towards developing comprehensive system understanding. In this paper, we decompose the ideal Quantum Teleportation protocol into a collection of simple quantum circuit blocks, examine the behavior of each block, and show how collections of blocks operate to create more complex circuits. We believe this approach greatly simplifies the understanding of how the Quantum Teleportation protocol works. This paper is introductory in nature and is intended to help those who are new to modeling, simulating, and analyzing ideal quantum circuits. Keywords: Bell State Generation (BSG), Bell State Measurement (BSM), Einstein Podolsky Rosen (EPR) pairs, quantum entanglement, superdense coding, quantum teleportation.

1 Introduction The evolution of quantum computing has driven the need for the development of quantum networks to interconnect geographically separated quantum computers [1,2]. The Quantum Teleportation protocol enables the transport of an arbitrary unknown quantum state from one location to another [3]. The goal of this paper is demonstrate how decomposing and abstracting the behavior of a complex system into a collection of smaller blocks can facilitate understanding of more complex behaviors. Specifically, we show how that decomposing the Quantum Teleportation protocol, an essential element of quantum networks, into its constituent blocks, studying the behavior of each block independently, and examining how interconnected collections of these blocks behave can simplify understanding of how the protocol works. The Quantum Teleportation protocol is often viewed as “magical” as it is the only way that one can transport an unknown quantum state from one location to another [2]. We seek to demystify this view to show that there is no “magic” behind the Quantum Teleportation protocol. One can easily understand how the protocol works by building a sound understanding of the mathematical abstractions of quantum mechanical blocks, examining the behavior of the constituent blocks, study the composition of collections of blocks, and by exercising simple mathematical analysis using college level algebra. In this paper, we assume the reader has a basic understanding of quantum information theory representations. The remainder of the paper is organized as follows. Section 2 offers a basic introduction of the abstract mathematical modeling of two-level quantum mechanical system, Section 3 discusses the Bell State Generation (BSG) block used to generate maximally entangled Bell state pairs, Section 4 discusses the Bell State Measurement (BSM) block used to measure in the Bell basis, Section 5 considers the behavior of the BSG block connected to the BSM block, Section 6 presents the Superdense Coding protocol, and Section 7 presents Quantum Teleportation protocol. Finally, Section 8 concludes the paper and proposes our future work.


7 Quantum Teleportation Protocol In this section, we introduce the Quantum Teleportation protocol which is the essential element used in quantum networks [8]. The Quantum Teleportation protocol was first proposed by Bennett, Brassard, Crépeau, Jozsa, Peres, and Wootters in 1993 [8]. It was first experimentally implemented in 1997 by two research groups: one led by Sandu Popescu and the other by Anton Zeilinger [9]. Experimental quantum teleportation has been demonstrated using photons, atoms, electrons, and superconducting circuits over distances up to 1,400 km by Jian-Wei Pan’s group using the Micius satellite for space-based quantum teleportation. The goal of the Quantum Teleportation protocol is to send an unknown quantum state |߰ ۧ ஼ from one point to another using a shared EPR pair and two classical bits. Fig. 11 shows the quantum gate diagram for Quantum Teleportation. The protocol accepts an unknown quantum state |߰ ۧ ஼ and uses the BSG block (Charlie), the BSM block (Alice), and the conditional X and Z gates (Bob). In this example, Charlie generates the Phi Plus |Ȱ ା ۧ ஺஻ maximally entangled Bell state. However, just as in the Superdense Coding discussed above, any of the three other maximally entangled Bell states can be used with appropriate adjustments in the circuit. In this circuit, Charlie shares half of the EPR pair with Alice and the other half with Bob. Alice conducts a Bell state measurement of the unknown quantum state and her half of the EPR pair which collapses the state and instantaneously communicates information to Bob’s half of the EPR pair. Alice communicates the outcomes of the BSM measurement, ܾ ଶ ܾ and ଵ , to Bob through a classical channel. Bob then applies the appropriate unitary transforms to his half of the EPR pair to recover the unknown quantum state. Bob applies a Pauli-Z operator when ܾ ଵ =1 and a Pauli-X operator when ܾ ଶ =1. The result of the protocol is that Bob has transformed his half of the EPR pair into |߰ ۧ ஼ ᇱ based upon the results of Alice’s Bell state measurement. Note that in our discussion we are presenting the ideal case when there are no errors or noise, ߰| so ۧ ஼ ߰|= ۧ ஼ ᇱ .

The key insight into understanding how the Quantum Teleportation protocol works is to understand that conducting a Bell state measurement forces the qubits that are input to the BSM block to collapse. Specifically, if the two qubits applied to the BSM block are part of a larger joint state, the measurement results of the BSM reveal which elements of the larger joint state still exist. In our view, this is the “secret sauce” in understanding the Quantum Teleportation protocol works. Once you grasp this concept, analysis of the rest of the protocol is simple algebraic manipulations. For this reason, it is useful to first rewrite qubits pairs in terms of the sums and differences of the four maximally entangled Bell states. When a Bell measurement occurs, it collapses the joint state in the given Bell basis. The results of the Bell state measurement reveal valuable information when the measured qubits are part of a larger joint state. Rewriting the terms in a joint state using this substitution is very similar to rewriting a state using a change of basis. Table 7 summarizes how you can rewrite a two qubit joint state as the sum or difference of two maximally entangled Bell states. It is not difficult to prove the equivalent representations shown in Table 7 and we encourage the reader to do so. Table 7 – Equivalent Representation of Two Qubit Joint States Input to a BSM using Sums and Differences of Bell States in The Bell Basis Two-Qubit Joint State Input to a BSM Equivalent Representation using Sums and Differences of Bell States |0 ۧ ٔ |0 ۧ =|00 ۧ 1 ξ 2 ( |Ȱ ା ۧ +|Ȱ ି ۧ) |0 ۧ ٔ |1 ۧ =|01 ۧ 1 ξ 2 ( |Ȳ ା ۧ +|Ȳ ି ۧ) |1 ۧ ٔ |0 ۧ =|10 ۧ 1 ξ 2 ( |Ȳ ା ۧ െ |Ȳ ି ۧ) |1 ۧ ٔ |1 ۧ =|11 ۧ 1 ξ 2 ( |Ȱ ା ۧ െ |Ȱ ି ۧ)

t is important to note that all three qubits shown in Eq. 20 are still in the same total state since no measurement has yet been performed. Instead, we have simply performed a change of basis in Alice’s part of the system. The actual teleportation occurs when Alice conducts her Bell state measurement and measures qubits in the Bell basis ( ߔ| ା ۧ ஼஺ ߔ|, ି ۧ ஼஺ ߖ|, ା ۧ ஼஺ ߖ|, ି ۧ ஼஺ ) . Alice’s local measurement of the joint state collapses two of the qubits into one of the four states with equal probability (25\%). After the Bell state measurement is completed, Alice sends the two classical bits resulting from the Bell state measurement to Bob over a classical channel informing him of the result. Bob’s half of the EPR pair, ߰| ۧ ஻ , instantaneously becomes one of the four corresponding superposition states shown in Table 8. The Bell state measurement causes the joint state to collapse leaving ߰| ۧ ஻ in a known state. Now, Bob simply has to perform a unitary transform to recover the original state based upon the two classical bits ܾ ଶ ܾ and ଵ resulting from the Bell state measurement that Alice sends him over a classical channel. Bob applies the transforms as shown in Table 9 to recover the state |࣒ ۧ ࡯ ᇱ . Note that the operators ܫ መ ܺ , ෠ , and ܼ መ are listed in the mathematical (not physical) order they are applied.

8 Conclusions and Future Work In this paper, we presented a demonstration of how decomposing a complex system into a series of smaller abstract functional blocks can be very helpful in developing comprehensive system understanding. Specifically, we identified and enumerated the basic quantum circuit blocks found in the ideal Quantum Teleportation protocol quantum circuit. We examined the mathematical models used to represent the behavior of each of the quantum circuit blocks and provided some general guidelines on how the blocks behave. We accomplished this by understanding what happens when the quantum basis states are operated upon by the corresponding mathematical operators. We found it is very useful to summarize and understand how each operator acts on the zero state, |0 ۧ , and the one state, |1 ۧ so that when you examine an new quantum circuit, you will have developed some intuition as to how the quantum circuit operates without the need to immediately calculate the model outputs. We then closely examined how the Bell State Generation (BSG) and Bell State Measurement (BSM) blocks operate to create and measure maximally entangled Bell states. We showed how developing a basic understanding of these blocks simplifies the analysis of quantum circuit that consists of a cascade of a BSG block with a BSM block. Next, built on this finding by introducing the Superdense Coding protocol and quantum circuit. We saw how adding the single qubit blocks ܺ ෠ ܼ and መ enabled Alice to change her half the EPR pair to transform the joint state seen by Bob to from Phi Plus |Ȱ ା ۧ ஺஻ to Psi Plus |Ȳ ା ۧ ஺஻ , Phi Minus |Ȱ ି ۧ ஺஻ , or Psi Minus |Ȳ ି ۧ ஺஻ . As a consequence, when Bob conducts his Bell state measurement, he will obtain the classical bits that Alice desired to send him using the quantum channel. At this point, we showed how quantum states can be manipulated to attain the goal of Alice sending two classical bits of information to Bob by changing her half of the EPR pair shared with Bob. Finally, we examined the Quantum Teleportation protocol and quantum circuit. We found that all of the functional blocks found in the Superdense Coding circuit were also present in Quantum Teleportation circuit, but with the addition of an unknown quantum state. We then presented the mathematical calculations required to understand how a three qubit joint state can be manipulated by a Bell state measurement. We showed the key insight into understanding how the Quantum Teleportation protocol works is to understand that conducting a Bell state measurement forces the qubits that are input to the BSM block to collapse. Specifically, if the two qubits applied to the BSM block are part of a larger joint state, the measurement results of the BSM reveal which elements of the larger joint state still exist. We then showed how Bob can recover the unknown quantum state by manipulating the joint quantum state using the operators ܫ መ ܺ , ෠ , and ܼ መ . Our hope is that this paper enables the reader to more easily gain a deeper insight and to develop a better understanding of how to analyze quantum circuit implementations of quantum protocols. While this information is introductory in nature, it can provide value to those who are new to modeling, simulating, and analyzing ideal quantum circuits.

