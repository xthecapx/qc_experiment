to reference: \cite{heinrichRandomizedBenchmarkingRandom_2023}

In its many variants, randomized benchmarking (RB) is a broadly used technique for assessing the quality of gate implementations on quantum computers. A detailed theoretical understanding and general guarantees exist for the functioning and interpretation of RB protocols if the gates under scrutiny are drawn uniformly at random from a compact group. In contrast, many practically attractive and scalable RB protocols implement random quantum circuits with local gates randomly drawn from some gate-set. Despite their abundance in practice, for those non-uniform RB protocols, general guarantees for gates from arbitrary compact groups under experimentally plausible assumptions are missing. In this work, we derive such guarantees for a large class of RB protocols for random circuits that we refer to as filtered RB. Prominent examples include linear cross-entropy benchmarking, character benchmarking, Pauli-noise tomography and variants of simultaneous RB. Building upon recent results for random circuits, we show that many relevant filtered RB schemes can be realized with random quantum circuits in linear depth, and we provide explicit small constants for common instances. We further derive general sample complexity bounds for filtered RB. We show filtered RB to be sample-efficient for several relevant groups, including protocols addressing higher-order cross-talk. Our theory for non-uniform filtered RB is, in principle, flexible enough to design new protocols for non-universal and analog quantum simulators.

INTRODUCTION 

Assessing the quality of quantum gate implementations is a crucial task in developing quantum computers [1, 2]. Arguably, the most widely employed protocols for this task are randomized benchmarking (RB) [3–8] and its many variants (see Ref. [9] for a recent overview) including linear cross-entropy benchmarking (XEB) [10]. The basic idea of RB is to measure the accuracy of random gate sequences of different lengths. Typically, this results in an experimental signal described by (a mixture of) exponential decays. Stronger noise results in faster decays with smaller decay parameters. Hence, those decay parameters are used to capture the average fidelity of the implemented quantum gates. A crucial advantage of these methods besides their experimental efficiency is that the reported decay parameters are robust against state preparation and measurement (SPAM) errors. Generally speaking, many experimental signatures can be rather well fitted by an exponential decay. Experimentally observing an exponential decay in an RB experiment does by itself not justify the interpretation of the decay parameter as a measure for the quality of the gates. In addition, RB requires a well-controlled theoretical model that explains the observed decays under realistic assumptions and provides the desired interpretation of the decay parameters. Extensive research has already established a solid theoretical foundation for RB, particularly when the gates comprising the sequences are drawn uniformly at random from a compact group. Generalizing the arguments of Refs. [8, 11–14], Helsen et al. [9] derived general guarantees for the signal form of the entire zoo of RB protocols with finite groups (which readily generalizes to compact groups [15]): If the noise of the gate implementation is sufficiently small, each decay parameter is associated to an irreducible representation (irrep) of the group generated by the gates. Thus, the decay parameter indeed quantifies the average deviation of the gate implementation from their ideal action on the subspace carrying the irrep. For example, the ‘standard’ RB protocol draws random multi-qubit gates uniformly from the Clifford group, except for the last gate of the sequence, which is supposed to restore the initial state. This protocol results in a single decay parameter associated with the irreducible action on traceless matrices and related to the average gate fidelity. In practice, however, the suitability of uniform RB protocols for holistically assessing the quality of noisy and intermediate-scale quantum (NISQ) hardware is restricted. On currently available hardware, sufficiently long sequences of multi-qubit Clifford unitaries lead to way too fast decays to be accurately estimated for already moderate qubit counts. More scalable RB protocols directly draw sequences of local random gates, implementing a random circuit [16–19]. We refer to those protocols that use a non-uniform probability distribution over a compact group as non-uniform RB protocols. Arguably, the most prominent example of non-uniform RB is the linear XEB protocol, which was used for the first demonstration of a quantum computational advantage in sampling tasks [10, 20].

Establishing theoretical guarantees for non-uniform RB is considerably more subtle. Roughly speaking, the interpretation of the decay parameter is more compliarXiv:2212.06181v3 [quant-ph] 27 Jun 202

2 cated as one additionally witnesses the convergence of the non-uniform distribution to the uniform one with the sequence length—causing a superimposed decay in the experimental data. These obstacles are well-known in the RB literature [17, 21] and have raised suspicion in the context of linear XEB [22, 23]. If not carefully considered, one easily ends up significantly overestimating the fidelity of the gate implementations. In the context of their universal randomized benchmarking framework, Chen et al. [24] have given a comprehensive analysis of non-uniform RB protocols using random circuits which form approximate unitary 2-designs. As such, the results in Refs. [24] are e.g. applicable to linear XEB with universal gate sets or with gates from the Clifford group [25]. The original theoretical analysis of linear XEB relies on the assumption that for every circuit, one observes an ideal implementation up to global depolarizing noise [10]. Building more trust in linear XEB has motivated a line of theoretical research, introducing different heuristic estimators [22] and analyzing the behaviour of different noise models in random circuits [26, 27] using mappings of random circuits to statistical models [28]. But general guarantees that work under minimal plausible assumptions on the gate implementation and for random circuits generating arbitrary compact groups—akin to the framework [9, 15] for uniform RB and going beyond unitary 2designs [24, 25] — are missing. Moreover, the sampling complexity of protocols like linear XEB for non-uniform distributions and general gate-dependent noise remains unclear. In this work, we close these gaps by developing a general theory of ‘filtered’ randomized benchmarking with random circuits using gates from arbitary compact groups under arbitrary gate-dependent (Markovian and timestationary) noise. Under minimal assumptions, we guarantee the functioning of the protocol, and give explicit bounds on sufficient sequence lengths as well as on the number of samples. Moreover, we specialize our general findings to concrete groups and random circuits, and give explicit constants. Concretely, the filtered RB protocol requires the execution of random circuit instances with a varying number of layers, c.f. Fig. 1. Deviating from standard RB, the last gate inverting the sequence is omitted and a simple computational basis measurement is performed instead. This approach simplifies the experimental procedure and is arguably a core requirement for experimentally scalable non-uniform RB. The inversion of the circuit is effectively performed in the classical post-processing of the data. At this stage, the experimental data is additionally filtered to show only the specific decay associated with a single irrep of the group generated by the random circuits. This step is the motivation for the name ‘filtered RB’ [9]. It is especially useful if the relevant group decomposes into many irreps which would otherwise result in multiple, overlapping decays.

Besides linear XEB, filtered RB [9] encompasses character benchmarking [29], matchgate benchmarking [30], and Pauli-noise tomography [31] as well as variants of simultaneous [32] and correlated [33] RB as additional examples. The filtering allows for a more fine-grained perspective on the perturbative argument at the heart of the framework of Ref. [9], in that the different irreps of a group can be analyzed individually. In this way, we derive new perturbative bounds based on the harmonic analysis of compact groups that can be naturally combined with results from the theory of random circuits. Thereby, we go significantly beyond previous works and treat uniform and non-uniform filtered RB on the same footing.

More precisely, our guarantees assume that the error of the average implementation (per irrep) of the gates appearing in the random circuit is sufficiently small compared to the spectral gap of the random circuit. Then, the signal of filtered RB is well-described by a suitable exponential decay after a sufficient circuit depth. The required depth depends inversely on the spectral gap and logarithmically on the dimension of the irrep. We show that for practically relevant examples, our results imply that a linear circuit depth in the number of qubits suffices for filtered RB. Furthermore, a sufficiently small average implementation error is ensured if the noise rate per gate scales reciprocally with the system size. Omitting the inversion gate comes at the price that the simple arguments for the sample-efficiency of standard randomized benchmarking do not longer apply to filtered RB. As in shadow estimation for quantum states [34], the post-processing introduces estimators that are generally only bounded exponentially in the number of qubits. Thus, the precise convergence of estimators calculated from polynomially many samples is a priori far from clear. Generalizing our perturbative analysis of the filtered RB signal to its variance, we derive general expressions for the sample complexity of filtered RB. In particular and under essentially the same assumptions that guarantee the signal form of the protocol, filtered RB is as sample-efficient as the analogous protocol that uses uniformly distributed unitaries. Again important examples are found to be already sample-efficient using linear circuit depth. Perhaps surprisingly, we find that filtered RB without entangling gates has constant sampling complexity independent of the non-trivial support of the irreps. This finding is in contrast to the related results in state shadow estimation. To showcase the general results, we explicitly discuss the cases where the random circuit generates the Clifford group, the local Clifford group, or the Pauli group. Moreover, we discuss common families of random circuits and summarize spectral gap bounds with explicit, small constants from the literature and our own considerations [35–39]. Finally, it is an open question whether the postprocessing of filtered RB can be modified so that meaningful decay constants can be extracted already from constant depth circuits. In the context of linear XEB, Ref. [22] introduces a heuristic so-called ‘unbiased’ estimator to this end. Using the general perspective of filtered RB, we sketch two general approaches to construct modified linear estimators for constant-depth circuits. The first approach introduces a more costly computational task in the classical post-processing. The second approach requires that the random distribution of circuits is locally invariant of local Clifford gates. We formally argue that these estimators work under the assumption of global depolarizing noise, putting them at least on comparable footing as existing theoretical proposals. A detailed perturbative analysis is left to future work. We expect that the theory of non-uniform filtered RB can be applied to many other practically relevant benchmarking schemes and bootstraps the development of new RB schemes. In fact, one of our main motivations for deriving the flexible theoretical framework is its applications for the characterization and benchmarking of non-universal and analog quantum computing devicesconsolidating and extending existing proposals [40, 41] in forthcoming and future work. On a technical level, we develop tools to analyze noisy random circuits using harmonic analysis on compact groups and matrix perturbation theory. We expect that this perturbative description also finds applications in quantum computing beyond the randomized benchmarking of quantum gates. The tools and results might, in principle, be applicable to analyze the noiserobustness of any scheme involving random circuits, e.g. randomized compiling [42], shadow tomography and randomized measurements [43] or error mitigation [44]. As a by-product, our variance bounds take a more direct representation-theoretic approach working with tensor powers of the adjoint representation rather than exploiting vector space isomorphisms and invoking Schur-Weyl duality [45]. This approach also opens up a complimentary, illuminating perspective on the sample-efficiency of estimation protocols based on random sequences of gates more generally. Prior and related work. Already one of the first RB proposals, NIST RB [7] classifies as non-uniform RB and was later thoroughly analyzed and compared to standard Clifford RB [21]. A first discussion of the obstacles arising from decays associated with the convergence to the uniform measure was then given in Ref. [21]. Further nonuniform RB protocols are approximate RB [16] and direct RB [17] (sometimes called generator RB). The original guarantees for these protocols rely on the closeness of the probability distribution to the uniform one in total variance distance, thus generally requiring long sequences. Direct RB ensures this closeness by starting from a random stabilizer state as the initial state—assumed noiseless in the analysis, which is additionally restricted to Pauli-noise. The restriction can be justified with randomized compiling [42, 46, 47], which essentially requires the perfect implementation of Pauli unitaries. After the publication of a preprint of this paper, a more thorough analysis of direct RB under general gate-dependent noise was published [48], using techniques which are similar to ours. The work by Helsen et al. [9] unifies and generalizes the guarantees for these RB protocols to gate-dependent noise but still works with convergence of the probability distribution to the uniform distribution in total variation distance. The approach of Ref. [9] extends previous arguments for the analysis of gate-dependent noise by Wallman [13] using the language of Fourier transforms of finite groups introduced to RB by Merkel et al. [14]. The argument straightforwardly carries over to compact groups [15]. The assumptions on the gate implementation required for the guarantees of Ref. [9], closeness in average diamond norm error over all irreps, are too strong to yield practical circuit depths for RB with random circuits.

This obstacle has been overcome in the universal randomized benchmarking framework by Chen et al. [24]. There, the authors are able to relax the assumption on the probability measure for the above protocols (“twirling schemes” in Ref. [24]) and only require that the channel twirl over this measure is within unit distance from the Haar-random channel twirl (in induced diamond norm or spectral norm). Hence, it is sufficient for these schemes to implement random circuits which form approximate unitary 2-designs w.r.t. the relevant norm. As such, it is necessary that the used distributions have support on groups which are unitary 2-designs, such as the unitary or the Clifford group [25]. Filtered RB, as formulated in Ref. [9], is a variant of character RB [29]. Linear XEB [10], when averaged over multiple circuits, can be seen as the special case of filtered RB when the group generated by the circuits is a unitary 2-design. Ref. [9] analyzes linear XEB, including variance bounds, but only for uniform measures, not for random circuits. Ref. [26] puts forward a different perturbative analysis for filtered randomized benchmarking schemes by carefully tracing the effect of individual Pauli-errors in random circuits. To our understanding, the argument, however, crucially relies on the heuristic estimator proposed in Ref. [22]. See also the review [20] for a detailed literature overview on linear XEB. Hybrid benchmarking [19] puts forward another approach to avoid the linear inversion using random Pauli observables. In contrast to other randomized benchmarking schemes, the hybrid benchmarking signal consists of linear combinations of (exponentially) many decays with complex poles [18, 19]. Estimating these poles, however, is typically infeasible, see the detailed discussion in Ref. [9]. The here discussed filtered RB protocols for circuits generating local groups is an alternative to simultaneous [32] and correlated [33] RB but is in addition capable of estimating higher-order correlations. A randomized benchmarking scheme with the Heisenberg-Weyl group is also proposed in Ref. [49]. Alternative approaches to filtered non-uniform RB aiming at better scalability of randomized benchmarking protocols are cycle RB [50, 51], average circuit eigenvalue sampling [52] and the recent RB with mirror circuits [53]. Outline. The remainder of this work is structured as follows: We start by introducing and discussing the filtered RB protocol in Sec. II. Afterwards, in Sec. III, we give a non-technical overview of our main results and highlight the central messages of this work. The technical part begins with Sec. IV, where we introduce necessary background and definitions. This section is self-contained and gives a general introduction into the techniques used in this paper. We then proceed by stating and proving our results in Sec. V. This section is structured into nine subsections which address the central assumptions of our work, the above described main results, some auxillary results, and the specialization to specific examples. In Sec. V I, we give a precise comparison to the technical assumptions and conclusions of related works. Finally the conclusion is given in Sec. VI. The appendices contain a summary of the relevant matrix perturbation theory in App. A, a discussion of singleshot versus multi-shot estimators in App. B, and a selfcontained computation of noise-free second moments for various groups in App. C.

CONCLUSION 

Filtered randomized benchmarking (RB) is the collection of experimental data obtained by applying random sequences of gates, followed by a suitable post-processing. Summarized by the motto ‘measure first – analyze later’, filtered RB is part of a modern class of characterization protocols such as shadow tomography [43, 45, 107], randomized gate set tomography [108, 109], and other random sequence protocols [54]. These protocols essentially share their first stage – the data acquisition from random sequences of gates – and only differ in the post-processing of this data. The advantage of such protocols is that different conclusions can be drawn from the same data. Compared to standard RB, filtered RB has the advantage to avoid the application of the final inversion gate. This avoids the problem that the inverse of a unitary can have large circuit depth even if the original unitary does not. Moreover, it relaxes the requirements on the used gate set as it is not needed to efficiently compute the inversion gate. Another advantage is only apparent when the protocol is used for groups G with more than one nontrivial irrep (i.e. groups which are not unitary 2-designs) – in this case, standard RB produces a linear combination of exponential decays in one-to-one correspondence with the irreps of G. In practice, fitting these decays is often not possible and there is no way of attributing them to individual irreps. Filtered RB allows to address specific irreps and produces a single exponential decay that is straightforward to fit.12 In this work, we have developed a general theory of filtered RB with random circuits and arbitrary gatedependent (Markovian and time-stationary) noise. Our theory is based on harmonic analysis on compact groups and neatly combines representation theory with the theory of random circuits. As such, it can be seen as a mathematically elegant advancement of Fourier-based approaches to RB [9, 14], which does not require to implement a Fourier transform ‘by hand’ but instead effectively performs it in the post-processing. We hope that our theory clarifies to which extent a group structure is needed for RB and how one can go ‘beyond groups’ [24]. Concretely, we have shown that if the implementation error of the used gates is small enough compared to the spectral gap of the random circuit, then the noise cannot close the spectral gap of the relevant moment operator. We argued that for local noise, this is the case if gate errors scale as O(1/n). As a consequence, the filtered RB signal has two well-defined contributions. The dominant one has the form of a matrix exponential decay and quantifies the average performance of the gate implementation. This decay is superimposed by a additional, subdominant decays reflecting the convergence of the random circuit to a 2-design for the group G. Importantly, if the implementation error is too large, the spectral gap may close and control over the contributions to the signal is lost. In this regime, the signal does generally not reflect the average circuit performance. We have derived sufficient conditions on the depth of the random circuit which guarantee that the subdominant contribution to the signal is negligible and the relevant decays can be extracted. For random circuits which mix sufficiently fast, a circuit depth which is at most linear in the number of qudits is sufficient. Although one may hope that this scaling can be improved to logarithmic depth, it is not clear whether this can be done under the general assumptions in this paper, or whether further assumptions on the physical noise (e.g. locality) is needed. Additionally, we have shown that the use of random circuits instead of uniformly drawn unitaries from G does not change the sampling complexity of filtered RB. In particular, filtered RB is sampling-efficient if it is sampling-efficient when uniformly distributed unitaries are used. To this end, we have computed the sampling complexity of ideal filtered RB for unitary 3-designs, local unitary 3-designs, and the Pauli group, and found that it is indeed sampling-efficient in these important cases. To illustrate our general results, we have applied them to commonly used groups and random circuits and have derived concrete, small constants for sufficient sequence lengths. These explicit computation may also serve as a guideline when applying our general results to other groups and random circuits. Finally, we have discussed other choices of filter functions which should result in a further reduction of the necessary circuit depths for filtered RB. Moreover, we think that these proposals are highly relevant for linear cross-entropy benchmarking (XEB) and the related random circuit sampling benchmark [26, 104]. However, a rigorous analysis for these alternative filter functions requires new techniques beyond the ones used in this paper and we leave such a study for future work. Ref. [54] showed that by using more general filter functions one can perform other robust gate-set characterization tasks, including “filtered” versions of interleaved randomized benchmarking, RB tomography, or Pauli channel estimation. Our techniques can be applied to analyse variants with non-uniform measures of these protocols. We note that our analysis can be extended to incorporate non-Markovian noise, we however leave such a treatise for future work.