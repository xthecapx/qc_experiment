\section{Conclusion}\label{sec:conclusions}

This article addresses the critical gap between hardware-level benchmarks and circuit-specific reliability by introducing a new framework. Combining a benchmarking strategy with a variation of the quantum teleportation protocol, our approach enables the identification of correlations between a circuit's pre-runtime characteristics and its post-runtime outcomes.

Beyond experimentation, the research was supported by theoretical analysis and simulations on classical hardware, providing a robust foundation for evaluating the algorithm execution performed on IBM Sherbrooke and Rigetti Ankaa-3 quantum devices.

Our findings suggest that during the NIQS era, quantum algorithm design should prioritize minimizing entangled qubit operations and circuit depth, as these factors negatively impact success. At the same time, we determine that single gate operations are not a concern if limited to fewer than 40 operations.

The major contribution of this work is a framework that systematically measures how pre-runtime metrics influence execution success on real quantum hardware. Specifically, it provides a method to quantify the influence that different circuit components have on overall reliability, offering developers practical, predictive insights into algorithm performance that go beyond traditional hardware-centric benchmarks.

\section{Future work}\label{sec:future_work}

The current study establishes a foundation for future research into a broader range of pre-runtime metrics capable of characterizing quantum circuits. A key objective will be to utilize these metrics to develop a robust, reduced-dimensionality "footprint" for quantum circuits. Such a footprint could enable classical hardware to forecast post-runtime metrics, thereby providing developers with valuable preliminary insights before QPU execution.

A practical objective is the development of a comprehensive toolkit to assist developers and researchers in capturing pre-runtime metrics and correlating them with post-runtime outcomes derived from various QPUs. Subsequent work will involve rigorously testing an expanded set of pre-runtime metrics to identify correlations with additional post-runtime characteristics, including execution time and computational cost.

Additionally, the vTP algorithm itself is designed with future extensibility in mind. As distributed quantum systems become available, the benchmarking framework and correlations established in this work can be directly adapted. Researchers and developers will be able to transition from the vTP to the canonical TP by replacing the fixed gate sequence with the conditional sequence of gates and the required measurements described by the protocol. This makes the methodology a future-proof strategy for benchmarking not only single QPUs today but also the quantum distributed system of tomorrow.

Finally, our long-term vision is to use the data gathered through this approach to develop artificial intelligence models. As a significant volume of data is acquired, future research will focus on creating machine learning models to forecast the behavior of quantum circuits, especially those whose complexity prevents them from being simulated using classical computational methods.

\section{AI Tools and Services Acknowledgment}

The authors acknowledge the use of AI tools and services for manuscript preparation and research assistance. Writefull.com supported language, grammar, and writing improvements; Elicit.com assisted in literature discovery; and Cursor helped with code development, including plotting scripts and quantum circuit understanding. These tools enhanced the authors' capabilities, while the core research, design, data interpretation, and conclusions are the authors' intellectual work.
